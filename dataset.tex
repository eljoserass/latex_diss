 
\documentclass[a4paper,10pt]{article}
\usepackage[utf8]{inputenc}

%opening
\title{}
\author{Jose Rodriguez}

\begin{document}

\maketitle

\begin{abstract}

\end{abstract}

\section{Dataset}
here some overview (why are we talking about qa, mathworld problems? -> because of the dataset (GSM8K))
\paragraph{Question and Answering Generation}
\paragraph{Math World problems}
A math word problem is a mathematical exercise where a significant part of the background information is presented in texts as natural language, rather than in mathematical notation (Part I ”Solving Arithmetic Mathematical Word Problems: A Review and Recent Advancements” Chandra et al. (2019)). These problems are part of the basic elementary school curriculum, starting with basic operations(addition, subtraction, multiplication, division), and as the students advance problems contain higher levels of complexity (e.g., rate, probability, permutation, combination). Solving this problem requires linguistic and reasoning comprehension, thus the study of how children solve them has been a challenging area of research in cognitive science and education psychology (Part I ”Solving Arithmetic Mathematical Word Problems: A Review and Recent Advancements” Chandra et al. (2019))). As solving them requires linguistic abilities and reasoning along with knowledge of arithmetic operations, researchers on AI and NLP have studied this as a task of building a system capable of replicating the cognitive process of solving these problems. 
\paragraph{Grade School Math 8k (GSM8K)}
\subsection{Evaluation}
here just about how the evaluation is done

\end{document}
